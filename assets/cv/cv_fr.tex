\documentclass[10pt]{article}

% This is a helpful package that puts math inside length specifications

\usepackage{calc}

\usepackage{amssymb}
\usepackage{amsmath,verbatim}

\usepackage[utf8]{inputenc}
% Layout: Puts the section titles on left side of page
\reversemarginpar

%
%         PAPER SIZE, PAGE NUMBER, AND DOCUMENT LAYOUT NOTES:
%
% The next \usepackage line changes the layout for CV style section
% headings as marginal notes. It also sets up the paper size as either
% letter or A4. By default, letter was used. If A4 paper is desired,
% comment out the letterpaper lines and uncomment the a4paper lines.
%
% As you can see, the margin widths and section title widths can be
% easily adjusted.
%
% ALSO: Notice that the includefoot option can be commented OUT in order
% to put the PAGE NUMBER *IN* the bottom margin. This will make the
% effective text area larger.
%
% IF YOU WISH TO REMOVE THE ``of LASTPAGE'' next to each page number,
% see the note about the +LP and -LP lines below. Comment out the +LP
% and uncomment the -LP.
%
% IF YOU WISH TO REMOVE PAGE NUMBERS, be sure that the includefoot line
% is uncommented and ALSO uncomment the \pagestyle{empty} a few lines
% below.
%

%% Use these lines for letter-sized paper
\usepackage[paper=letterpaper,
            %includefoot, % Uncomment to put page number above margin
            marginparwidth=1.2in,     % Length of section titles
            marginparsep=.05in,       % Space between titles and text
            margin=1in,               % 1 inch margins
            includemp]{geometry}

%% Use these lines for A4-sized paper
%\usepackage[paper=a4paper,
%            %includefoot, % Uncomment to put page number above margin
%            marginparwidth=30.5mm,    % Length of section titles
%            marginparsep=1.5mm,       % Space between titles and text
%            margin=25mm,              % 25mm margins
%            includemp]{geometry}

%% More layout: Get rid of indenting throughout entire document
\setlength{\parindent}{0in}

%% This gives us fun enumeration environments. compactitem will be nice.
\usepackage{paralist}

%% Reference the last page in the page number
%
% NOTE: comment the +LP line and uncomment the -LP line to have page
%       numbers without the ``of ##'' last page reference)
%
% NOTE: uncomment the \pagestyle{empty} line to get rid of all page
%       numbers (make sure includefoot is commented out above)
%
\usepackage{fancyhdr,lastpage}
\pagestyle{fancy}
%\pagestyle{empty}      % Uncomment this to get rid of page numbers
\fancyhf{}\renewcommand{\headrulewidth}{0pt}
\fancyfootoffset{\marginparsep+\marginparwidth}
\newlength{\footpageshift}
\setlength{\footpageshift}
          {0.5\textwidth+0.5\marginparsep+0.5\marginparwidth-2in}
\lfoot{\hspace{\footpageshift}%
       \parbox{4in}{\, \hfill %
                    \arabic{page} of \protect\pageref*{LastPage} % +LP
%                    \arabic{page}                               % -LP
                    \hfill \,}}

% Finally, give us PDF bookmarks
\usepackage{color,hyperref,graphicx}
\definecolor{darkblue}{rgb}{0.0,0.0,0.3}
\hypersetup{colorlinks,breaklinks,
            linkcolor=darkblue,urlcolor=darkblue,
            anchorcolor=darkblue,citecolor=darkblue}

%%%%%%%%%%%%%%%%%%%%%%%% End Document Setup %%%%%%%%%%%%%%%%%%%%%%%%%%%%


%%%%%%%%%%%%%%%%%%%%%%%%%%% Helper Commands %%%%%%%%%%%%%%%%%%%%%%%%%%%%

% The title (name) with a horizontal rule under it
%
% Usage: \makeheading{name}
%
% Place at top of document. It should be the first thing.
\newcommand{\makeheading}[1]%
        {\hspace*{-\marginparsep minus \marginparwidth}%
         \begin{minipage}[t]{\textwidth+\marginparwidth+\marginparsep}%
                {\large \bfseries #1}\\[-0.15\baselineskip]%
                 \rule{\columnwidth}{1pt}
                 
         \end{minipage}}

% The section headings
%
% Usage: \section{section name}
%
% Follow this section IMMEDIATELY with the first line of the section
% text. Do not put whitespace in between. That is, do this:
%
%       \section{My Information}
%       Here is my information.
%
% and NOT this:
%
%       \section{My Information}
%
%       Here is my information.
%
% Otherwise the top of the section header will not line up with the top
% of the section. Of course, using a single comment character (%) on
% empty lines allows for the function of the first example with the
% readability of the second example.
\renewcommand{\section}[2]%
        {\pagebreak[2]\vspace{1.3\baselineskip}%
         \phantomsection\addcontentsline{toc}{section}{#1}%
         \hspace{0in}%
         \marginpar{
         \raggedright \scshape #1}#2}

% An itemize-style list with lots of space between items
\newenvironment{outerlist}[1][\enskip\textbullet]%
        {\begin{itemize}[#1]}{\end{itemize}%
         \vspace{-.6\baselineskip}}

% An environment IDENTICAL to outerlist that has better pre-list spacing
% when used as the first thing in a \section 
\newenvironment{lonelist}[1][\enskip\textbullet]%
        {\vspace{-\baselineskip}\begin{list}{#1}{%
        \setlength{\partopsep}{0pt}%
        \setlength{\topsep}{0pt}}}
        {\end{list}\vspace{-.6\baselineskip}}

% An itemize-style list with little space between items
\newenvironment{innerlist}[1][\enskip\textbullet]%
        {\begin{compactitem}[#1]}{\end{compactitem}}

% To add some paragraph space between lines.
% This also tells LaTeX to preferably break a page on one of these gaps
% if there is a needed pagebreak nearby.
\newcommand{\blankline}{\quad\pagebreak[2]}

%\newcommand{\photo}[3][1=64pt,2=0.4pt,usedefault]{\def\@photowidth{#1}\def\@photoframewidth{#2}\def\@photo{#3}}% the 1st (optional) argument is the width of the photo, the 2nd (optional) argument is the thickness of the frame around it.

%%%%%%%%%%%%%%%%%%%%%%%% End Helper Commands %%%%%%%%%%%%%%%%%%%%%%%%%%%

%%%%%%%%%%%%%%%%%%%%%%%%% Begin CV Document %%%%%%%%%%%%%%%%%%%%%%%%%%%%

\begin{document}

\pagestyle{empty}
%\vspace{-5cm}

%\begin{figure}
%
%\vspace{-1cm}
%\flushright\includegraphics[scale=0.3]{picture2}
%\end{figure}


\makeheading{Dr. Ing.~Marcelo FORETS IRURTIA}



\section{Informaci\'{o}n de contacto}

%
% NOTE: Mind where the & separators and \\ breaks are in the following
%       table.
%
% ALSO: \rcollength is the width of the right column of the table 
%       (adjust it to your liking; default is 1.85in).
%
\newlength{\rcollength}\setlength{\rcollength}{3.5in}%
%


\begin{tabular}[t]{@{}p{\textwidth-\rcollength}p{\rcollength}}
Oficina: & Bureau 209, \href{http://www-verimag.imag.fr/}{VERIMAG} – \href{https://www.univ-grenoble-alpes.fr/}{Universit\'e Grenoble Alpes} \\
Direcci\'on:& \href{https://batiment.imag.fr/en}{Bâtiment IMAG}, 700 Av. Centrale, 38041, France \\
P\'agina web: & \href{http://marcelo-forets.fr}{\url{http://marcelo-forets.fr}}\\
Tel\'efono m\'obil: & (+33)(0)6-47-91-43-65 \\
E-mail personal: & \href{mailto:mforets@gmail.com}{mforets@gmail.com} \\
E-mail trabajo: & \href{mailto:marcelo.forets-irurtia@univ-grenoble-alpes.fr}{marcelo.forets-irurtia@univ-grenoble-alpes.fr} \\
Nacionalidad: & Uruguaya \\
Edad y situaci\'on  familiar: & 29, Casado
\end{tabular}

%\section{}
%
%\begin{tabular}[t]{@{}p{\textwidth-\rcollength}p{\rcollength}}
%2 Rue Alphonse Allais& \textit{T\'el:} (+33) 9-50-66-19-27\\
%38400 St. Martin d’H\`eres, France    & \textit{E-mail:}
%\href{mailto:mforets@gmail.com}{mforets@gmail.com}\\
%Nationalit\'e: & Uruguayenne \\
%\^Age et situation familiale: & 27, Mari\'e
%\end{tabular}

\rule{\textwidth}{0.5pt}

\section{Cargo actual}
\href{http://www.ujf-grenoble.fr/}{\textbf{Universit\'e Grenoble Alpes}}, Grenoble, France
\hfill\textbf{Enero 2016-actual}
\begin{itemize}
\item[-] Investigador de Post-Doctorado en el grupo TEMPO.
\item[-] Palabras clave: sistemas híbridos, verificación formal, sistemas dinámicos no lineales, métodos composicionales, teoría de control robusto, optimización matemática, computación científica.
\end{itemize}

%\section{Temas de inter\'es acad\'emico}
%\begin{itemize}
%\item[-] Physique math\'{e}matique: th\`eorie spectrale des \'{e}quations d'evolution,\\ analyse numérique, syst\`emes dynamiques.
%
%\item[-] Physique th\'{e}orique: simulation quantique, percolation, th\'{e}orie des champs en espace courbe.
%
%\end{itemize}

\section{Formaci\'{o}n}
%

%\blankline

\href{http://www.ujf-grenoble.fr/}{\textbf{Universit\'e Joseph Fourier}}, Grenoble, France
\hfill\textbf{2013--2015}

\begin{outerlist}
\item[-] Doctorat en Math\'ematiques et Informatique à l'\'Ecole Doctorale MSTII
	\begin{innerlist}
        \item[-] Laboratorio: \href{http://www.liglab.fr/}{Laboratoire d'Informatique de Grenoble}, \'equipe \href{http://capp.imag.fr/wordpress/}{CAPP}.
        \item[-] Título de la tesis: Marches Quantiques et M\'ecanique Quantique Relativiste.
	\item[-] Directores de tesis: \href{http://membres-lig.imag.fr/arrighi/}{Pablo Arrighi} (directeur) et \href{http://www-fourier.ujf-grenoble.fr/~joye/}{Alain Joye} (co-encadrant).
	\item[-] Palabras clave: mecánica cuántica discreta, simulación de sistemas cuánticos en interacción, análisis espectral, caminatas cuánticas, mecánica cuántica relativista.
	\end{innerlist}


\end{outerlist}

\blankline

\href{http://www.fing.edu.uy/}{\textbf{UdelaR}}, Facultad de Ingeniería, Montevideo, Uruguay
\hfill\textbf{2006-2012}

\begin{outerlist}

\item[-] Ingeniero Electricista, opci\'on Electr\'onica, 2012 
\item[-] Tesis de grado: \href{http://iie.fing.edu.uy/investigacion/grupos/lai/}{CubeSatET:} Diseño e implementación de una estación espacial terrestre como datalink para un satélite remoto.

\end{outerlist}

\blankline

\href{http://www.fing.edu.uy/}{\textbf{UdelaR}}, Facultad de Ciencias, Montevideo, Uruguay
\hfill\textbf{2006-2010}

\begin{outerlist}

	\item[-] Licenciado en F\'isica, opci\'on F\'isica (\emph{Licence en Physique})
	\item[-] Clasificación: 10.04/12 (1er rango por año de obtención de diploma)

\end{outerlist}

\section{Idiomas}

espa\~nol (lengua materna), ingl\'es (corriente), franc\'es (corriente), portugu\'es (nociones)

%\blankline
%
%\href{http://www.fing.edu.uy/}{\textbf{UdelaR}}, Facultad de Ciencias, Montevideo, Uruguay
%\hfill\textbf{2006-2010}
%
%\begin{outerlist}
%	
%	\item[-] Licenciado en F\'isica, opci\'on F\'isica (\emph{Licence en Physique})
%	\item[-] Clasificación: 10.04/12 (1er rango por año de obtención de diploma)
%	
%\end{outerlist}

\section{Experiencia profesional}

\href{http://www.ujf-grenoble.fr/}{\textbf{Univ. Grenoble Alpes}}, Grenoble, France
\hfill\textbf{2016--2017}

\begin{outerlist}
	\item[]\textit{Investigación \& Proyectos}
	\begin{innerlist}
		\item[-] NANO2017. Estudio de la variabilidad de circuitos electrónicos analógicos. Colaboración con ST Microelectronics.
	\end{innerlist}
\end{outerlist}

\blankline

\href{http://www.ujf-grenoble.fr/}{\textbf{Univ. Joseph Fourier}}, Grenoble, France
\hfill\textbf{2013--2015}

\begin{outerlist}
\item[]\textit{Investigación \& Proyectos}
\begin{innerlist}
\item[-] Doctorante en el equipo CAPP del Laboratorio de Informática de Grenoble, y del equipo de \href{http://www-fourier.ujf-grenoble.fr/?q=fr/content/physique-mathematique}{Física Matemática} del Instituto Fourier.
\end{innerlist}
\end{outerlist}

\blankline

\href{http://www.udelar.edu.uy}{\textbf{UdelaR}}, Montevideo, Uruguay 
\hfill\textbf{2008--2012}

\begin{outerlist}
\item[]\textit{Enseñanza}
\begin{innerlist}
\item[-] \href{http://www.fing.edu.uy}{\textbf{Facultad de Ingenier\'ia}}: Docente (ayudante, 20hs/sem) para cursos de 1er y 2o año de las carreras de Ingeniería (física general, mecánica, electromagnetismo), y de la Licenciatura en Física (mecánica estadística, laboratorio).
\item[-] \href{http://www.fhuce.edu.uy/}{\textbf{Facultad de Humanidades y Ciencias de la Educaci\'{o}n}}: Docente del curso de Probabilidad y Estadística.
\end{innerlist}

\item[]\textit{Investigación \& Proyectos}
\begin{innerlist}
\item[-] Miembro del equipo ``Física Computacional y Mecánica Estadística" del Instituto de Física. Palabras clave: procesamiento cuántico de la información, correlaciones cuánticas, algorithms de búsqueda. 
\item[-] Miembro del equpo ``Plataforma de simulaci\'{o}n para el sistema de energ\'{i}a el\'{e}ctrico'' \href{http://iie.fing.edu.uy/simsee/}{SimSEE} del Instituto de Ingeiería Eléctrica (IIE), proyecto ANII. Palabras clave: optimización estocástica, programación no lineal.
\end{innerlist}

\end{outerlist}

%dddd (agregar el de humanidades) Cours de Probabilit\'e pour les Sciences Sociales

%\section{Distinctions obtenues}
%
%\href{http://www.xxxxx}{\textbf{Universit\`e Joseph Fourier, ED-MSTII}}, Grenoble, France
%\hfill\textbf{octobre 2013}
%
%\begin{outerlist}
%\item[] Allocations president. 
%%	\begin{innerlist}
%%        \item Laboratoire : \href{http://www.liglab.fr/}{Laboratoire d'Informatique de Grenoble}, \'equipe \href{http://capp.imag.fr/wordpress/}{CAPP} (Calcul, Algorithmes, Programmes et Preuves)
%%        \item Titre : Automates cellulaires quantiques et simulation de systèmes physiques
%%	\item Directeurs: \href{http://membres-lig.imag.fr/arrighi/}{Pablo Arrighi} (directeur) et \href{http://www-fourier.ujf-grenoble.fr/~joye/}{Alain Joye} (co-encadrant)
%%	\item Mots-cl\'es : automates cellulaires quantiques, m\`ecanique quantique discr\`ete, simulation quantique, analyse des EDPs, analyse spectrale
%%	\end{innerlist}
%
%
%\end{outerlist}
%
%\blankline
%
%\href{http://www.xxxxx/}{\textbf{Scientific Research Commision (CSIC)}}, Montevideo, Uruguay
%\hfill\textbf{2011-2012}
%
%\begin{outerlist}
%
%\item [] Scholarship directed to enseignants from University, awarded to 24 out of 78 postulants.
%%
%%        \begin{innerlist}
%%        \item Grade : 7.8/12     
%%        \item Projet de th\'ese (avec P. Cabalo et S. Alpuy) : \href{http://iie.fing.edu.uy/investigacion/grupos/lai/}{CubeSatET:} Design and implementation of a terrestrial station to provide datalink with a remote satellite.
%%        \item Directeur: Juan Pechiar 
%%        \end{innerlist}
%%
%%\item[] Licence en Physique, 2010 
%%	\begin{innerlist}
%%        \item Grade : 10/12 
%%        \item Cours approfondis : Computation quantique, Th\`eorie de Nombres, Computation de haute performance, xxx. 
%%        \end{innerlist}
%
%\end{outerlist}
%
%\blankline
%
%\href{http://www.xxxxx/}{\textbf{Research initiation scholarship (PEDECIBA)}}, Montevideo, Uruguay
%\hfill\textbf{2008-2009}
%
%\begin{outerlist}
%
%\item [] Scholarship directed to enseignants from University, awarded to 24 out of 78 postulants.
%%
%%        \begin{innerlist}
%%        \item Grade : 7.8/12     
%%        \item Projet de th\'ese (avec P. Cabalo et S. Alpuy) : \href{http://iie.fing.edu.uy/investigacion/grupos/lai/}{CubeSatET:} Design and implementation of a terrestrial station to provide datalink with a remote satellite.
%%        \item Directeur: Juan Pechiar 
%%        \end{innerlist}
%%
%%\item[] Licence en Physique, 2010 
%%	\begin{innerlist}
%%        \item Grade : 10/12 
%%        \item Cours approfondis : Computation quantique, Th\`eorie de Nombres, Computation de haute performance, xxx. 
%%        \end{innerlist}
%
%\end{outerlist}
%
%

\newenvironment{benumerate}[1]{
    \let\oldItem\item
    \def\item{\addtocounter{enumi}{-2}\oldItem}
    \begin{enumerate}
    \setcounter{enumi}{#1}
    \addtocounter{enumi}{1}
}{
    \end{enumerate}
}

\section{Publicaciones en revistas con revisón por pares (selección)}

\begin{outerlist}
%\item[] 
%\begin{benumerate}{4}
%\item Enhancing entropy production via dynamical percolation (joint with R. Donangelo, A. Romanelli). \emph{(unpublished)}

\item[-] Quantum Walking in Curved Spacetime, Pablo Arrighi, Stefano Facchini, M. F. \emph{Quantum Information Processing} (2016) 15: 3467.

\item[-] \href{http://iopscience.iop.org/1367-2630/16/9/093007/}{Discrete Lorentz covariance for Quantum Walks and Quantum Cellular Automata}. P. Arrighi, S. Facchini. \emph{New Journal of Physics}, 16 (2014) 093007. 

\item[-] \href{http://arxiv.org/abs/1307.3524}{The Dirac equation as a quantum walk: higher dimensions, observational convergence}. P. Arrighi, V. Nesme and M.F. \emph{Journal of Physics A: Mathematical and Theoretical}, 47 (2014) 465302.

%\item \href{http://journals.cambridge.org/action/displayAbstract?fromPage=online&aid=8544748&fileId=S0960129511000600}{Spatial quantum search in a triangular network} (joint with G. Abal, R. Donangelo and R. Portugal). Publi\'{e} en \emph{Mathematical Structures in Computer Science}, 22 , pp 521-531, Cambridge University Press. Voir aussi arXiv pre-print: \href{http://arxiv.org/abs/1009.1422}{arXiv:1009.1422}
%\end{benumerate}
\end{outerlist}

\section{Publicaciones en conferencias con revisión por pares  (selección)}

\begin{outerlist}
%	\begin{benumerate}{3}
	
		\item[-] Modeling the Wind Turbine Benchmark with PWA Hybrid Automata. Applied Verification for Continuous and Hybrid Systems (ARCH 2017). Palabras clave: reachability analysis, compositional methods, nonlinear control systems, SpaceEx. 

		\item[-] Constructing Verification Models of Nonlinear Simulink Systems via Syntactic Hybridization. Nikolaos Kekatos, M. F., Goran Frehse. 56th IEEE Conference on Decision and Control, to be held in Melbourne, Australia (2017). 
		
		\item[-] Semidefinite Characterization of Invariant Measures for Polynomial Systems. Victor Magron, M. F., Didier Henrion. 18th French-German-Italian conference on Optimization, en Paderborn, Germany (2017). Palabras clave: Keywords: invariant measures, dynamical systems, polynomial optimization, semidefinite programming, moment-sum-of-square relaxations, Christoffel function. 

%	\end{benumerate}
\end{outerlist}


\section{Trabajos submitidos en proceso de revisión}

\begin{outerlist}
	\item[] 
	\begin{benumerate}{3}

		\item Reach Set Approximation through Decomposition with Low-dimensional Sets and High-dimensional Matrices, Sergiy Bogomolov, M. F., Goran Frehse, Andreas Podelski, Christian Schilling, Frédéric Viry. 2017. Palabras clave: reachability analysis, safety verification, linear time-invariant systems, set recurrence relation.
		
		\item Occupation measure methods for modelling and analysis of biological hybrid automata. 2017. Thao Dang, Eric Fanchon, M. F., Victor Magron, Alexandre Rocca. arXiv pre-print: \href{https://arxiv.org/abs/1710.03158}{arXiv:1710.03158} Palabras clave: biological modelling, hybrid dynamical system, optimal control problem, semidefinite optimization, occupation measures. 
		
		\item Explicit Error Bounds for Carleman Linearization. 2017. M. F., Amaury Pouly. arXiv pre-print: \href{https://arxiv.org/abs/1711.02552}{arXiv:1711.02552} Palabras clave: carleman linearization, polynomial ODEs, infinite-dimensional systems, guaranteed integration, nonlinear control theory.  

	\end{benumerate}
\end{outerlist}

\section{Presentaciones orales}

\begin{itemize}

	\item[-]  Quantum walking in curved spacetime: (3+1) dimensions, and beyond , 2016. 7th colloquium of the CNRS GDR Quantum Engineering, Foundations and Applications (IQFA). 

	\item[-]  Journ\'{e}es Informatique Quantique 2013, Nancy, France (2013). \emph{The Dirac Quantum Walk}.

	\item[-] Mathematics, Statistics and Applied Mathematics Seminar, NUI Galway, Ireland (2014). Orador invitado. \emph{The Cauchy problem for the continuous limit of Quantum Walks}. 

\end{itemize}

\section{Presentaciones en formato póster  y seminarios (selección)}

\begin{outerlist}
	%	\begin{benumerate}{3}
	
	\item[-] Colloque du \href{http://gdriqfa.unice.fr/?lang=en}{GDR IQFA}, Lyon, France (2014). Poster: \emph{Discrete Lorentz covariance}.
	
	\item[-] Quantum walks and quantum simulations, Scuola Normale Superiore, Pisa, Italy (2013). Poster: \emph{The Dirac equation as a Quantum Walk}.
	
	%	\end{benumerate}
\end{outerlist}


\section{Participación en Conferencias y Workshops} 

\begin{itemize}

\item[-] \href{http://www.fis.puc.cl/~smp2015/frontpage.html}{Summer School on current topics in Mathematical Physics}, Valparaiso, Chile, (2015). Lecturers: Barry Simon, Gunther Uhlmann, Simone Warzel, Jakob Yngvason.

\item[-] \href{http://www.math.kit.edu/iana3/page/isem/en}{18th ISEM 2014/2015}, Blaubeuren, Germany (2015). Form Methods for Evolution Equations, and Applications.

\item[-] \href{http://www.lebesgue.fr/content/sem2015-math-phy}{Meeting in Mathematical Physics}, Nantes, France, (2015). Institut Lebesgue, en partenariat avec \href{http://www.math.sciences.univ-nantes.fr/NOSEVOL/node/2}{ANR Nosevol}, et \href{http://www.math.sciences.univ-nantes.fr/dynqua/Accueil}{GDR Dynamique quantique}.

\item[-] \href{http://pillet.univ-tln.fr/aoqs/www/}{Advances in Open Quantum Systems}, Autrans, France (2013). Organizado por el proyecto \href{http://bruneau.u-cergy.fr/ANR/ANR.html}{ANR HamMark}.

\item[-] III Workshop-Escola de computação e Informação Quântica. Brazilian National Laboratory of Computer Science (LNCC), Petrópolis, RJ, Brasil (2010).

\item[-] NASA \& NIA 2010 RASC-AL Design Competition Forum (2010). 

\item[-] II Quantum Information School \& Workshop, Paraty, RJ, Brasil (2009).

\end{itemize}


%\item[] 
%\begin{itemize}
%\item \href{http://iopscience.iop.org/1367-2630/16/9/093007/}{Discrete Lorentz covariance for Quantum Walks and Quantum Cellular Automata} (joint with P. Arrighi, S. Facchini). Publi\'{e} en \emph{New Journal of Physics}, 16 (2014) 093007. Voir aussi arXiv pre-print: \href{http://arxiv.org/abs/1404.4499}{arXiv:1404.4499}
%\item \href{http://arxiv.org/abs/1307.3524}{The Dirac equation as a quantum walk: higher dimensions, observational convergence} (joint with P. Arrighi, V. Nesme). Publi\'{e} en  \emph{Journal of Physics A: Mathematical and Theoretical}. Voir aussi arXiv pre-print: \href{http://arxiv.org/abs/1307.3524}{arXiv:1307.3524}
%\item \href{http://journals.cambridge.org/action/displayAbstract?fromPage=online&aid=8544748&fileId=S0960129511000600}{Spatial quantum search in a triangular network} (joint with G. Abal, R. Donangelo and R. Portugal). Publi\'{e} en \emph{Mathematical Structures in Computer Science}, 22 , pp 521-531, Cambridge University Press. Voir aussi arXiv pre-print: \href{http://arxiv.org/abs/1009.1422}{arXiv:1009.1422}
%\end{itemize}
%\end{outerlist}

\section{Organización de Conferencias y Workshops}

\begin{itemize}

	\item[-] \href{http://gm-rqw.imag.fr/}{Meeting in Relativistic Quantum Walks}, Grenoble, France, (2014). Organización científica y material de la conferencia.

	\item[-] II Reuni\'{o}n conjunta de la Sociedad Argentina de F\'{i}sica y de la Sociedad Uruguaya de F\'{i}sica, Montevideo, Uruguay (2011). Miembro del comité organizador.
\end{itemize}

\section{Distinciones} 
\begin{itemize}
	\item[-] Beca de doctorado de la Univ. Joseph Fourier, Grenoble, Francia (2013-2015).
\item[-] Finalizaci\'{o}n de estudios de posgrado para docentes de la UdelaR. CSIC (2011). 
\item[-] Beca de Iniciación a la Investigación. PEDECIBA, Uruguay (2008). G. Abal, R. Donangelo.  
\end{itemize}

\section{Actividades Académicas}
\begin{itemize}
\item[-] Hybrid Systems: Computation and Control (HSCC) Repeatability Evaluation Commitee Member (2016, 2017).
\item[-] 20th International Federation of Automatic Control
(IFAC) World Conference, Reviewer, 2017. 
\item[-] Revisor en otras revistas científicas y conferencias: Computational Methods in Systems Biology (CMSB), Hybrid Systems Biology (HSB), Physica A: Statistical Mechanics and its Applications, Nature Scientific Reports, IEEE/ACM Transactions on Computational Biology and Bioinformatics.
\item[-] Miembro del jurado de tesis de grado en la Univ. ORT, Uruguay (2012) y en la Univ. de la Rep\'{u}blica, Uruguay (2010). 
\item[-] Delegado estudiante del comit\'e de PEDECIBA F\'isica (2011-2012).
\end{itemize}

%\section{Idiomas}
%
%espa\~nol (lengua materna), ingl\'es (corriente), franc\'es (corriente), portugu\'es (nociones)

\section{Otras actividades}

\begin{itemize}
	\item[-] Miembro del comit\'e organizador de la Olimpiada Nacional de Fisica, Uruguay (2009-actual), y delegado de Uruguay en las OIbF en Granada, España (2013).
	\item[-] Desarrollador open-source del software de cálculo matemático SageMath. 
\end{itemize}

\section{Conocimiento Técnico}

\begin{itemize}
	\item[-] Herramientas de productividad. Sistemas de preparación de documentos: LaTeX. Sistemas operativos: Linux, MacOSX. Sistemas de control de revisión: git
	\item[-] Lenguages de programación científica: Julia (experto), Python/NumPy/SciPy (experto), MATLAB (experto), Cython (intermedio). Lenguages de programación de sistemas: C (intermedio), Fortran (nociones), Java (nociones).
\end{itemize}
 

%\section{Loisirs}
%
%voyager, histoire des sciences, math\`ematique recreationnelle

\end{document}

%%%%%%%%%%%%%%%%%%%%%%%%%% End CV Document %%%%%%%%%%%%%%%%%%%%%%%%%%%%%



%\begin{itemize}
%	\item[-] Hybrid Systems: Computation and Control (HSCC) Repeatability Evaluation Commitee Member (2016, 2017).
%	\item[-] 20th International Federation of Automatic Control
%	(IFAC) World Conference, Reviewer, 2017. 
%	\item[-] Revisor en varias revistas científicas y conferencias: Computational Methods in Systems Biology (CMSB), Hybrid Systems Biology (HSB), Physica A: Statistical Mechanics and its Applications, Nature Scientific Reports, IEEE/ACM Transactions on Computational Biology and Bioinformatics.
%	\item[-] Miembro del jurado de tesis de grado: Univ. ORT, Uruguay (2012); Univ. de la Rep\'{u}blica, Uruguay (2010). 
%	\item[-] Miembro del comit\'e organizador de la Olimpiada Nacional de Fisica, Uruguay (2009-actual), y delegado de Uruguay en las OIbF en Granada, España (2013).
%	\item[-] Delegado estudiante del comit\'e de PEDECIBA F\'isica (2011-2012).
%\end{itemize}







